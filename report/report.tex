\documentclass{scrreprt}					% siehe <http://www.komascript.de>
\usepackage[english]{babel}				% Das Beispieldokument ist in Deutsch,
                								% daher wird mit Hilfe des babel-Pakets
               								% über Option ngerman auf deutsche Begriffe
               								% und gleichzeitig Trennmuster nach den
               								% aktuellen Rechtschreiberegeln umgeschaltet.
               								% Alternativen und weitere Sprachen sind
               								% verfügbar (siehe <http://ctan.org/pkg/babel>).
\begin{document}

% ----------------------------------------------------------------------------
% Titel (erst nach \begin{document}, damit babel bereits voll aktiv ist:
\titlehead{University of Applied Sciences Hamburg}% optional
\subject{Lab report}							% optional
\title{Machine learning}					% obligatorisch
%\subtitle{hi i bims}						% optional
\author{Marc Gehring, 2266937}			% obligatorisch
\date{Dec 2020}								% sinnvoll
\publishers{Prof. Dr. S. Hallerberg}	% optional
\maketitle

% verwendet die zuvor gemachte Angaben zur Gestaltung eines Titels
% ----------------------------------------------------------------------------

% Inhaltsverzeichnis:
\tableofcontents
% ----------------------------------------------------------------------------

% Gliederung und Text:
\chapter{Linear Classifiers}
\section{Linear Regression}
\label{sec:linregression}
Linear regression is great. Good stuff. The best. You'll see
\section{Bayesian Classifier}
\label{sec:bayesian}
ra ra ra ra ra ra ra ra
\section{Support Vector Machines}
\label{sec:svm}
support vector machineeeees
\chapter{Neural Networks}
\section{Introduction}
\section{MNIST Dataset}

\end{document}

